\section{Introduction}
Android applications have become a staple of many people's everyday mobile computing user experience.  As of the first quarter of 2017 Android possessed 85\% of the mobile phone market \cite{chau_2017}.  The need for complete and thorough testing capabilities of Android applications has never been greater. There is currently a deficit in an area of testing for mobile devices regarding the "state" of the mobile device while testing is taking place. Very often, particular states will cause a failure of a mobile application and currently there are not any good tools available to test different states to avoid these types of failures.

The state of the device is the current settings of different modules within the mobile device such as if bluetooth is on or off or if airplane mode is on or off; other "states" include having different applications running that access common instrumentation or having hardware devices connected to the mobile device and running with an application such as medical Electrocardiograms.  Any configuration of these different instruments, applications, operating system configurations, or modules can be considered a device's state; hereto referred as DS for device state.

We have developed a novel approach to very easily run existing Android test scripts against any DS one could imagine.  The solution we built is called "TADS" (Testing Application to Device State) that uses Espresso, an automated Android application testing tool, to test the Mobile application against multiple DSs.  TADS is primarily concerned with identifying errors caused by changing device states and not as concerned with identifying why the state change caused a failure.

Espresso is a tool developed by Google that enables a user to easily record a series of events, then create an oracle; which, is a way of knowing if the series of events executed correctly, so that the user can create an automated test case.  The Espresso recorder autmatically generates a testing script that can be executed at any time to recreate the series of events and check the oracle to ensure the series of events executed correctly.  The script then generates a pass or fail message that the user can see to know if any recent development work broke the existing software application.

TADS uses a simple principle of looping through various DSs.  TADS will set a baseline DS, run the Espresso script, get the results, then change the DS, and then re-run the Espresso script and report those results.  A conglomuration of testing different DSs becomes a function in the TADS application called a Device State Change hereto referred as a DSC.  One example is the bluetooth DSC.  Bluetooth is turned on by TADS, the Espresso script is executed, then bluetooth is turned off by TADS, and lastly the Espress script is executed and the results of that test are reported to the user.

It is more expensive for a company to not test device states against DSs because the failures are often catastrophic.  Companies have encountered these bugs after releasing an update for their application on the users' devices and the failure often causes a loss of credibility, loss of functionality that must then be quickly corrected and updated, and ultimately costs the company a loss of customers.  Testing the different states of just an application has been enhanced with tools such as Espresso, Barista, and Robotium \cite{optimusinformationinc2016} but the lack of testing those application states against the device state is where the problem lies.  Since the state of an application can be easily saved and the state of a device can be easily changed programmatically, a solution that automatically tests application states with different device states is of great value.  For applications being used in different device states, such as a "mapping" applications used outside of network range or medical applications using sensors for monitoring functions, application to device state testing becomes an extremely important issue.  Some of these issues can even be life safety critical.  One example is that of medical applications utilizing portable Electrocardiograms.  These devices are used to monitor a patient's heart rythems in order to detect anomolyes that are symptoms of particular life threatening diseases \cite{medical eval paper}.  If a DS causes a small failure in the medical application, incorrect results can be reported resulting in a false positive or a false negative.  The innovative ways applications are used with new hardware is only increasing.  State testing of applications with new hardware devices is a critical area of research to ensure that applications can continue to operate well. 

An emerging field that is gaining tremendous traction in the mobile device industry where testing of DSs is absolutly critical is that of context aware applications.
According to Luo et al, there is a strong need for innovative approaches to testing context aware applications \cite{Luo:2017:TLT:3139486.3130945}.  A "context aware" application is an application that is able to detect information about the device's physical environment using instrumentation and then do something with that information.  For instance, a context aware application can get GPS information and see that a user is at work, get location information from the network to see where in the building the user is, then get information from the calendar application about what is scheduled at that moment, and then the context aware application will determine that the user is in a board room in their office building during a scheduled meeting, and puts the phone on silent automatically or even stops all calls and replies with a text message saying the user is unavailable. DS testing is critical for these types of applications.  If another application acccessing certain instrumentation information collides with the context aware applications access of that data, the context aware application will fail. Currently, there is not a way to determine if that will happen using common automated testing tools. 

As of the writing of this paper, there is not any research available about automatic device state testing.  Our work highlights and addresses this need directly and gives researchers and companies an option to pursue that enables them to create DS testing with their Android applications.

TADS also tackles a further research problem, presented by Fazzini et al. \cite{7927971}, of testing instrumentation in an automated fashion.  They highlight the fact that most instrumentation and DS testing against an application is currently being done manually and that it is very time consuming and expensive because of the personnel hours required to complete that testing. They created Barista which is a better way to record and execute Android testing than Espresso.  Barista is an application that records user interactions with any Android application and automatically can generate oracles and then it records those interactions and oracles into an Espresso type script for later execution.  TADS can be easily integrated with Barista to enable the user to run Espresso scripts generated by Barista with different DSs.

TADS demonstrates that a library for doing state testing of devices that can be run with Espresso scripts is highly beneficial and easy to make.  This research contributes a tool that can be used right now for anyone wanting to improve their test suites to include device state testing. The full source code can be seen and downloaded at \url{https://github.com/UCCS-CS5371-Fall2017/jsander7/tree/master/ProjectFiles}.

The second contribution this work makes is a very simple way for testing context-aware applications' handling of instrumentation state changes. The work done by Luo et al \cite{Luo:2017:TLT:3139486.3130945} produces test data for context applications.  TADS tests the affect of changing the state of the instruments used by those applications.  A DSC can be easily generated that changes the states of the instruments that the context aware application uses in order to ensure changing instrumentation states will not cause a failure or worse, a false positive or false negative.

\section{Background}
TADS is built out of the Microsoft Power Shell application.  TADS also utilizes the Android Device Bridge and Android Studio.  The details of those are presented below.

Android Studio is Google's integrated development environment for developing Android applications.  Android studio generates a project when an Android application is started and within that project is a testing project.  When a developer makes an automated test those scripts are included in that test project.  Then, when the application is installed on a mobile device with the testing project attached, the test scripts are included on the device and can be executed via the Android Device Bridge.  

NEEDED because outlined above?? - Espresso, an android testing application, has an excellent tool for testing Mobile application UI's \cite{nolan2015agile}.  The tool records what is happening at a code level while a user performs different actions using the UI.  This enables a working application to have a test automatically generated that can then be run at a later time which can be used to create a regression test suite.  This actually enables states of the application to be tested without having to use any form of state machines or modeling.  This tool called "Test Recorder" is foundational to TADS. 

"Android Debug Bridge aka ADB, is a command-line utility included with Google's Android sdk. ADB can control your device over USB from a computer, copy files back and forth, install and un-install apps, run shell commands, and more." \cite{hoffman2017}  TADS relies heavily on the ADB. The ADB allows scripts to be run from a system other than the device itself either over a hard wire connection such as a USB connection from a computer to the device or over a network connection.  For some tests that TADS runs, a hardwire connection is required because at times the network connection is being tested and a hardwire connection is required when the network is turned off.  

TADS is developed in Microsoft Powershell.  Poweshell is an enhanced version of the MSDos program.  Powershell includes the .NET runtime and libraries which enables a developer to leverege the tools that .NET provides.  There is a Powershell integrated Development Environment called Windows Powershell ISE.  With the ISE all of the results of executing an ADB command can be viewed and recorded if desired.  Powershell also has the ability to organize code into modules and plain Powershell scripting files.  The modules are suffixed with a file type of .psm1.  The normal script files are suffixed with the file type .ps1.  TADS is made up of many functions in several modules and and Powershell files that execute ADB commands from a computer that the device is connected to.   


\section{Related Work}
There are several tools developed that can capture various information of an application and store said information for later tests.  These tools can be leveraged to create interesting app states instead of simple object states.  For instance Paulovsky et al. \cite{7962332} built a tool that automatically captures UI information as a user utilizes an application. 

Many others, such as Fazzini et al \cite{7927971} have made tools for recording tests that can be later executed in ways that are platform independent which is a nice utility that we will not be concerned with in this work.  Others have made unit level state testing models such as MilaniFard el al. \cite{MilaniFard:2014:LET:2642937.2642991} in which the state of the application is tested by automatically building a model using a dynamic and static crawler and then running that created model against a verification algorithm.  Their work levereges data modeling to find issues device state whereas TADS utilizes real world scenarios executed on real world devices.  

G. Bai et al developed a way to use model testing to find security vulnerabilities \cite{7911333}.  Choi et al have used machine learning to find the states of an application that are probably of value to test that have not been tested and make a model of those states for testing \cite{Choi:2013:GGT:2544173.2509552}.  Their work could be applied by developers utilizing TADS to develop DSCs appropriate for their application type.


\section{Implementation}
The main purpose of this Application is to simplify testing of different device states with the application.  There are preloaded "device state changes" (DSCs) that can be used in a test suite to test the app states that are chosen for testing.  A DSC is a test case that starts with a device state then runs the test cases and then changes the Devices state.  The DSC then runs the tests again with the new state set.  A few examples of DSCs are: airplane mode off, run a test suite or test case, switch airplane mode to on, then re-run a test suite or test case.

All of the files are available to be seen and modified.  New DSCs can be easily added using the idioms of the TADS solution in order to make very thorough test cases.

The TADS solution is built on Microsoft Power Shell and written as Power Shell scripts.  Power Shell was decided on to enforce DRY and SOLID principles without the convolution of building an executable program that is compiled.

The TADS solution is run by executing a powershell script.  There are global variables that must be populated by the user to indicate several items.  Figure 2. demonstrates those global variables.  They are modified in the file called "Project Main.ps1"  The easiest way to execute TADS testing is via the Powershell ISE. 
make figure 2., explain how it all works together...what funcs are on what files and why.

Table 1 is a break down of the functions to be called by your command line or in a script that a developer would write to test their application how they deem necessary.  

There is a readme file that should be followed to get the testing environment set up for usage of the TADS solution.  The TADS solution can be downloaded at: \url{https://github.com/UCCS-CS5371-Fall2017/jsander7/tree/master/ProjectFiles} \\

\begin{figure}[t]
	\centering
	\caption[Public Interface]{Public Interface}
	\label{fig:table1}
	\includegraphics[width=1\linewidth]{table1}
\end{figure}

The implementation allows a user to call a function to test the following as is shown in Table 1.  The user can run all of their test suites and all of the DSC's available, all tests and 1 DSC that is selected, 1 DSC and 1 test case in the suite, and all DSCs and 1 test case. The name of the application should be the whole app name starting with "com.". The public interface in Table 1 is what novice programmers should use.   

All functions in the entire solution are available to be consumed by the user.  This is so that a programmer can build their own Powershell script out of the existing functions.  

Within the TADS solution there are three main files.  They are DSCs.psm1, TestSuiteCommands.psm1, and ADBCommands.psm1.  The DSCs file contains all of the DSCs which is comprised of calls to the TestSuiteCommands and ADBCommands.  The TestSuiteCommands file contains the public interface found in Figure 1 and the commands that run the test suites in Espresso.  The ADMCommands file is where the state change commands are located that use ADB to change the states of the device. 

\section{Evaluation}
This is gibberish because I still am unsure how to evaluate this project.  Lorem ipsum dolor sit amet, consectetur adipiscing elit. Quisque laoreet facilisis dolor sit amet iaculis. Sed in ante lacinia purus accumsan commodo. Sed ut purus ante. Integer sed dictum urna. Vivamus vehicula quam sed urna placerat, at egestas dui malesuada. Vestibulum lectus urna, congue at malesuada at, pellentesque in mauris. Duis quam turpis, euismod nec lobortis sit amet, pretium non felis. Donec interdum purus in mi lacinia, ornare egestas nibh varius. Etiam semper elementum dolor nec malesuada. Pellentesque vestibulum vehicula velit ut fermentum. Donec mi eros, consequat vitae feugiat eget, commodo feugiat sem. Duis ultricies, justo quis faucibus accumsan, erat nisi cursus mi, nec imperdiet justo leo id nulla. Nunc at sodales turpis.

Sed non sem ac odio varius efficitur eget quis dui. Fusce mattis eros non convallis molestie. Fusce at mi dictum, cursus mauris vel, interdum erat. Vivamus fermentum lectus tristique dui varius interdum vel at enim. Vestibulum nisl turpis, consequat at quam sed, vulputate suscipit lectus. Duis et placerat odio, sed sollicitudin mi. Cras eleifend vulputate fermentum. Class aptent taciti sociosqu ad litora torquent per conubia nostra, per inceptos himenaeos. Etiam efficitur ullamcorper tincidunt. Suspendisse augue enim, commodo ut felis ut, accumsan rutrum mi. Mauris varius ornare vulputate. Proin neque leo, pulvinar id varius vel, fringilla nec dolor. Duis facilisis maximus metus nec rutrum.

Aliquam rutrum dui ipsum, quis egestas ante dapibus a. Maecenas eleifend, elit non ullamcorper imperdiet, odio enim congue leo, nec consequat nibh magna eget mi. Nulla dignissim ipsum ac tortor molestie, sit amet pulvinar odio semper. Vestibulum lobortis velit nec enim finibus, et accumsan nisi porta. Phasellus tincidunt malesuada ultrices. Nunc a neque quis sem elementum semper vel eu arcu. Donec vitae varius lacus. Nullam id urna maximus, consectetur ligula a, faucibus massa.

Aenean egestas ut nisl quis vehicula. Duis euismod bibendum felis in porta. Ut ut nisl at felis aliquam lacinia id eget ipsum. Nunc auctor dictum aliquet. Praesent sed orci nunc. Quisque placerat enim quis egestas pellentesque. Vivamus egestas dui dolor. Pellentesque at lectus non mauris viverra pellentesque eget et leo. Pellentesque luctus felis id laoreet euismod. Vestibulum vel eros nec velit tincidunt pellentesque. Quisque vitae malesuada augue. Suspendisse et vestibulum nulla.

Aliquam volutpat felis interdum risus laoreet pulvinar vitae at purus. Donec mauris sapien, placerat quis euismod vel, hendrerit at metus. Suspendisse vitae odio sit amet orci viverra interdum eu ut urna. Morbi porta tincidunt odio, eleifend finibus urna pulvinar sed. Fusce egestas nulla sed dui imperdiet, tempor sagittis felis rhoncus. Mauris elementum sed ex eget sollicitudin. Nullam malesuada, sem non porttitor fringilla, risus diam mattis enim, ut sollicitudin magna sapien et sem. Vivamus tempus vel nibh non mattis.

\section{Future Research}
A future research opportunity would be to implement the TADS solution in a way that the state changes can be called easily within an Espresso script.  That would enable knowledgeable testers to correctly evaluate the affect of different state changes on their solutions at specific times during execution of the script.   

Another challenge is that commands do not exist for ADB to change proprietary device instrumentation states.  Figuring out how to instantiate different hardware devices from the command interface and easily implement new DSCs will prove extremely beneficial in making robust state testing of Android applications using proprietary instrumentation suchh as a portable Electrocardiograms.  

Barista \cite{7927971} provides a better script generation tool than Espresso.  Integrating TADS to run Barista could prove to be a valuable endeavor.  Also, being able to inject DSCs into Barista generated testing scripts will enhance the ability of Barista to catch bugs introduced by device state changes.  

We also hope to expand the instrumentation capabilities to include wearable devices such as Android watch or even google glass for state testing with apps running on those devices.

Another approach that could prove extremely beneficial is to evaluate the affect of these sttes and state changes on the efficiency of an application. One could create a way of capturing pertinent throughput data and run some form of an evaluative algorithm against that date thereby giving developers pertinent information on how to improve their application's efficiency by using some kind of brownout technique \cite{Klein:2014:BBM:2568225.2568227} or other viable solution when that device state is detected. 

The last future research we are considering at this time is to investigate how device state affects installs.  Installations can be affected by the state of different items running on a device and building a test script that installs an application and running that installation on devices with different states may prove valuable; though, some research into whether or not industry struggles with installations being affected by state would be an interesting approach.  

ADD table with dsc's and the name of them with a note in the caption stating more would be added if actually publishing.  


\section{Threats to Validity}
One serious threat to this work is that the automated test scripts built in Espresso can be affected by state changes \cite{7927971} which can cause a failure in the test due to the script's failure and not the application's.  This error will be revealed by the logging utility and corrections can be made a test design time.  

